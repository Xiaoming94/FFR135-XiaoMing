\documentclass{article}

\usepackage[utf8]{inputenc}
\usepackage[margin=1in]{geometry}

\title{Code used during HW1 Problem1}
\begin{document}
\maketitle
\begin{verbatim}
# hopfieldnetwork.py
# author: Henry Yang (940503-1056)

import numpy as np
import random

###############################################
# Class for the Hopfield Model Neural Network #
# Using McCulloch Pitts Neurons               #
###############################################

def genRandPatterns(patterns,bitts):
    pat = np.random.randint(2,size=(patterns,bitts)) * 2 - 1
    return pat

def calcWeights(patterns):
    (_,bitts) = patterns.shape 
    w = np.zeros((bitts,bitts))
    for p in patterns :
        vecP = p.reshape(bitts,1)
        w = w + vecP @ np.transpose(vecP)


    return 1/bitts * w


class HopField(object):
    def __init__(self,usedata=False,data=None,patterns=10,bitts=5):
        # Generating Patterns
        if not usedata :
            pat = genRandPatterns(patterns,bitts)

        else :
            pat = data

        # Storing The Patterns
        self.patterns = pat

        # Storing the weights
        self.weights = calcWeights(pat)

    # Asynchronus feed Update
    def feedAsync(self,input,neuronIndex):
        #(bitts,_) = self.weights.shape
        #neuronIndex = np.random.randint(low=0,high=bitts)
        neuronWeights = self.weights[neuronIndex,:]
        z = neuronWeights @ input
        neuronState = np.sign(z)
        if neuronState == 0:
            neuronState = 1
        return neuronState
    
    # Synchronous feed.
    # Implemented as a matrix multiplication
    def feedSync(self,input):
        z_vec = self.weights @ input
        neuronStates = np.sign(z_vec)
        neuronStates[np.where(neuronStates == 0)] = 1
        return neuronStates


# problem1b.py
# Author: Henry Yang (940503-1056)

import numpy as np
from hopfieldnetwork import HopField

#########################################################
# This is the source file for solving the first second  #
# part of problem 1 in Assignment1 of FFR135            #
# The parameters chosen are given by the Assignnment    #
#########################################################

patterns = [12,20,40,60,80,100]
trials = 10 ** 5
bitts = 100
p_errors_list = []

for p in patterns :
    errors = 0
    for i in range(trials):
        net = HopField(patterns=p,bitts=bitts)
          
        # Randomly chooose a pattern and which neuron to observe
        n_i = np.random.randint(bitts)
        p_i = np.random.randint(p)
        inputPattern = net.patterns[p_i]
        
        neuronState = net.feedAsync(inputPattern,n_i)
        if(neuronState != inputPattern[n_i]):
            errors += 1
    
    p_error = errors/trials
    print("p_error : %s , for %s patterns" % (p_error,p))

    p_errors_list.append(p_error)

print(p_errors_list)

        
# problem1.py
# Author: Henry Yang(940503-1056)

import numpy as np
from hopfieldnetwork import HopField

#########################################################
# This is the source file for solving the first part of #
# problem 1 in Assignment1 of FFR135                    #
# The parameters chosen are given by the Assignnment    #
#########################################################

##
# Setting Parameters
patterns = [12,20,40,60,80,100] # Pattern counts to test
trials = 10 ** 5    # Number of Independent trials
bitts = 100         # Number of bitts in each pattern
p_errors_list = []

# Performing the simulation

for p in patterns :
    errors = 0
    for i in range(trials):
        net = HopField(patterns=p,bitts=bitts)

        # Set Diagonal to Zeroes
        for i in range(bitts):
            net.weights[i,i] = 0

        # Randomly chooose a pattern and which neuron to observe
        n_i = np.random.randint(bitts)
        p_i = np.random.randint(p)
        inputPattern = net.patterns[p_i]
        
        # Feed the pattern and check the results
        # Feeding the pattern using asynchronous update
        neuronState = net.feedAsync(inputPattern,n_i)
        if(neuronState != inputPattern[n_i]):
            errors += 1
    
    p_error = errors/trials
    print("p_error : %s , for %s patterns" % (p_error,p))

    p_errors_list.append(p_error)

print(p_errors_list)
\end{verbatim}
\end{document}
