\begin{verbatim}

# perceptron.py
# Author: Henry Yang (940503-1056)

import numpy as np

c_error_t = 0.12
###############################################################
# Classfile for two layer perceptron.                         #
# Implements Stochastic gradient descent with Backpropagation #
# Slightliy different from lecture slides                     # 
# All instances of this network got two inputs and one output #
############################################################### 

class TLPerceptron :
    def __init__(self,size_m1,size_m2,learning_rate):
        # Initialising Weight matrix input -> Hidden1
        self.weight_jk = np.random.rand(size_m1,2) * 2 - 1

        # Initialising Weight Matrix Hidden1 -> Hidden2
        self.weight_ij = np.random.rand(size_m2,size_m1) * 2 - 1

        # Initialising Weight Matrix Hidden2 -> output
        self.weight_i = np.random.rand(size_m2) * 2 - 1

        # Initialising Threshold1
        self.threshold_j = np.random.rand(size_m1) * 2 - 1
        
        # initialising Threshold2
        self.threshold_i = np.random.rand(size_m2) * 2 - 1

        # initialising Output Threshold
        self.threshold_o = np.random.rand() * 2 - 1

        self.learning_rate = learning_rate

    def feed(self,input):
        states = {}
        state_vj = np.tanh(self.weight_jk @ input - self.threshold_j)
        states['vj'] = state_vj
        state_vi = np.tanh(self.weight_ij @ state_vj - self.threshold_i)
        states['vi'] = state_vi
        output = np.tanh(self.weight_i @ state_vi - self.threshold_o)
        return (output, states)
    
    def just_feed(self,input):
        output, _ = self.feed(input)
        return output

    def train(self, training_set, validation_set, batch_size = 1):
        val_energy_arr = []
        train_energy_arr = []
        c_errors = []
        c_error = 1
        num_data,_ = np.shape(training_set)
        #valdata_num,_ = np.shape(validation_set)
        iterations = 0
        while c_error > c_error_t:
            if iterations % 1000 == 0:
                train_energy = self.calc_energy(training_set)
                train_energy_arr.append(train_energy)
                val_energy = self.calc_energy(validation_set)
                val_energy_arr.append(val_energy)
                c_error = self.calc_cerror(validation_set)
                print("on iteration %s, energy: %s, c_error: %s " % (iterations,val_energy,c_error))
                c_errors.append(c_error)
            
            mu_i = np.random.randint(num_data)
            tdata_row = training_set[mu_i,:]
            t_point = tdata_row[0:2]
            t_label = tdata_row[-1]
            output,states = self.feed(t_point)

            update_vals = self.backpropagation(t_point, output, t_label, states )
            self.update_network(update_vals)

            iterations += 1
            
        return c_errors, train_energy_arr, val_energy_arr

    def backpropagation(self, input , output, target ,states):
        update_vals = {}
        delta = -1 * ( target - output ) * (1 - output ** 2)
        update_vals['dw_i'] = delta * states['vi']
        update_vals['dto'] = delta * (-1)
        delta = delta * self.weight_i * (1 - (states['vi'] ** 2))
        update_vals['dw_ij'] = delta.reshape(np.size(states['vi']),1) @ states['vj'].reshape(1,np.size(states['vj']))
        update_vals['dti'] = delta * (-1)
        delta = (np.transpose(self.weight_ij) @ delta) * (1 - states['vj'] ** 2)
        update_vals['dw_jk'] = delta.reshape(np.size(states['vj']),1) @ input.reshape(1,2)
        update_vals['dtj'] = delta * (-1)

        return update_vals

    # Gradient descent
    def update_network(self, update_vals):
        # Updating weights

        self.weight_i = self.weight_i - self.learning_rate * update_vals['dw_i']
        self.weight_ij = self.weight_ij - self.learning_rate * update_vals['dw_ij']
        self.weight_jk = self.weight_jk - self.learning_rate * update_vals['dw_jk']

        # Update Thresholds
        self.threshold_o = self.threshold_o - self.learning_rate * update_vals['dto']
        self.threshold_i = self.threshold_i - self.learning_rate * update_vals['dti']
        self.threshold_j = self.threshold_j - self.learning_rate * update_vals['dtj']
    
    def calc_cerror(self, validation_set):
        val_in = validation_set[:,0:2]
        val_t = validation_set[:,-1]

        output = np.array([self.just_feed(vi) for vi in val_in])
        return 1/(2 * np.size(val_t)) * np.sum(np.abs(np.sign(output) - val_t))

    def calc_energy(self, validation_set):
        val_in = validation_set[:,0:2]
        val_t = validation_set[:,-1]

        output = np.array([self.just_feed(vi) for vi in val_in])
        return 1/2 * np.sum((val_t - output) ** 2)

# problem2.py
# Author: Henry Yang (940503-1056)

import numpy as np
import os
from singlelayernetwork import LCNetwork

learning_rate = 0.02

def test_linear_separability(targets, input_data): 
    for i in range(10):
        net = LCNetwork(input_length)
        net.train(input_data,targets,learning_rate)
        outputs = np.array([net.feed(p) for p in input_data])
        error_count = np.sum(np.abs(np.sign(outputs)-targets))
        if error_count == 0:
            return True
    
    return False

input_data = np.genfromtxt(os.path.join('.','input_data_numeric.csv'),delimiter=',')
_,input_length = np.shape(input_data[:,1:])

targets_mat = np.array([[-1, -1, -1, 1, -1, 1, 1, 1, 1, 1, 1, -1, 1, -1, -1, -1]
         ,[1, 1, 1, -1, 1, 1, 1, 1, -1, -1, -1, -1, 1, 1, -1, -1]
         ,[1, -1, 1, -1, -1, 1, -1, -1, -1, -1, -1, -1, 1, -1, -1, 1]
         ,[1, 1, 1, 1, 1, 1, -1, -1, 1, 1, 1, 1, 1, -1, 1, -1]
         ,[1, 1, 1, -1, -1, 1, -1, -1, -1, 1, -1, -1, -1, -1, -1, -1]
         ,[-1, 1, 1, -1, -1, 1, -1, -1, -1, -1, 1, 1, -1, 1, -1, -1]
         ])

separable_results = []

for targets in targets_mat:
    separable = test_linear_separability(targets, input_data[:,1:])
    separable_results.append(separable)


for (i,separable) in zip(list(range(len(separable_results))),separable_results):
    if separable:
        print("function %s is linearly separable" % i)
    else:
        print("function %s is not linearly separable" % i)import numpy as np


# problem4.py

import os
from perceptron import TLPerceptron
import matplotlib.pyplot as plt

size_m1 = 16
size_m2 = 8

training_set_path = os.path.join('.','training_set.csv')
validation_set_path = os.path.join('.','validation_set.csv')

training_set = np.genfromtxt(training_set_path,delimiter = ',')
validation_set = np.genfromtxt(validation_set_path,delimiter = ',')

training_points = training_set[:,0:2]
validation_points = validation_set[:,0:2]

training_targets = training_set[:,-1]
validation_targets = validation_set[:,-1]

train_true = training_points[np.where(training_targets == 1)]
train_false = training_points[np.where(training_targets == -1)]

val_true = validation_points[np.where(validation_targets == 1)]
val_false = validation_points[np.where(validation_targets == -1)]

#plt.figure()
#pf = plt.scatter(train_false[:,0],train_false[:,1],c='r',marker='o')
#pt = plt.scatter(train_true[:,0],train_true[:,1],c='b',marker='o')
#plt.title("Training data distribution")
#plt.legend([pf,pt],["negative","positive"])
#plt.figure()
#pf = plt.scatter(val_false[:,0],val_false[:,1],c='m',marker='o')
#pt = plt.scatter(val_true[:,0],val_true[:,1],c='g', marker='o')
#plt.title("validation data distribution")
#plt.legend([pf,pt],["negative","positive"])
#plt.show()

net = TLPerceptron(size_m1,size_m2,0.01)

c_errors, train_energy_arr, val_energy_arr = net.train(training_set,validation_set)

print(net.weight_jk.shape)

np.savetxt(os.path.join('.','w1.csv'),net.weight_jk,delimiter=',')
np.savetxt(os.path.join('.','w2.csv'),net.weight_ij,delimiter=',')
np.savetxt(os.path.join('.','w3.csv'),net.weight_i,delimiter=',')

np.savetxt(os.path.join('.','t1.csv'),net.threshold_j,delimiter=',')
np.savetxt(os.path.join('.','t2.csv'),net.threshold_i,delimiter=',')
#np.savetxt(os.path.join('.','t3.csv'),np.array(net.threshold_o),delimiter=',')

plt.figure()
train_line, = plt.plot(np.arange(len(train_energy_arr)),train_energy_arr,'b-')
val_line, = plt.plot(np.arange(len(val_energy_arr)),val_energy_arr,'g-')
plt.legend([train_line, val_line],["Training Energy","Validation Energy"])
plt.grid(b=True)
plt.title("Energy overtime")

measure_count = 40
x = np.linspace(-1,1,measure_count);
y = np.linspace(-1,1,measure_count);

xv,yv = np.meshgrid(x,y)

measure_points = np.transpose(np.array([xv.reshape(-1),yv.reshape(-1)])) 
#print(measure_points)
results = np.array([np.sign(net.just_feed(xp)) for xp in measure_points])
pp = measure_points[np.where(results == 1)]
np = measure_points[np.where(results == -1)]

plt.figure()
plt.scatter(val_false[:,0],val_false[:,1],c="m",marker='o')
plt.scatter(val_true[:,0],val_true[:,1],c="g",marker='o')
plt.scatter(np[:,0],np[:,1],c="black",marker='+',s=10)
plt.scatter(pp[:,0],pp[:,1],c="cyan",marker='+',s=10)
plt.show()

print(net.threshold_o)# singlelayernetwork.py
# Author: Henry Yang (940503-1056)

import numpy as np

iterations = 10 ** 5

class LCNetwork:
    def __init__(self,i_n):
        self.inputs = i_n
        self.weights = np.random.uniform(-0.2,0.2,i_n)
        self.threshold = np.random.rand() * 2 - 1
    
    def feed(self,input):
        return np.tanh((1/2) * self.weights @ input - self.threshold)
    
    def train(self,data, targets, learning_rate):
        energy_arr = []
        indices = np.random.randint(2 ** self.inputs,size=iterations)
        iteration = 0
        for i in indices:
            if iteration % 100 == 0:
                print("energy after %s iterations: %s" % (iteration,self.calc_energy(data,targets)))
            energy_arr.append(self.calc_energy(data,targets))
            sample = data[i]
            output = self.feed(sample)
            target = targets[i]
            dw_i, dt = self.backprop(sample,output,target,)
            self.weights -= learning_rate * dw_i
            self.threshold -= learning_rate * dt
            iteration += 1
        return energy_arr

    def backprop(self,input,output, target):
        delta = -1*(target - output) * (1 - output ** 2)
        dw_i = delta * input
        dt = delta * (-1)
        return dw_i, dt
    
    def calc_energy(self,data,targets):
        outputs = np.array([self.feed(x) for x in data])
        return (1/2) * np.sum((targets - outputs)**2)

\end{verbatim}